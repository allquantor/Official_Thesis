\chapter{Evaluation of Results: Value for Business}\label{Chapter:6}
Since the confusion matrices for all predictive models are available (see Chapter~\ref{Chapter:5:Results}), it is possible to analyze the potential monetary impact of a model on the micro-lending business. In this chapter, the value of a model for business is calculated.

Please note that some financial indicators used in calculations below are only approximate but in the range of real figures of a micro-lending company. For business-related reasons, the exact figures cannot be reported in this work.

The business value calculations utilize the following characteristics:

\[\textrm{(Cost of Customer Acquisition (marketing cost))} \; CAC = 0.5  \]
\[\textrm{(Mean value of the interest rate)}\; r = 10\%  \]
\[\textrm{(Mean value of a loan taken by a legitimate applicant)}\; C_p = 550  \]
\[\textrm{(Mean value of a loan taken by a fraudster)}\; C_n = 2.100  \]


First, the marketing costs for \textrm{all} negative (fraud) cases are calculated:
\[ Mcost_N = CAC * |N| \]

Once a malicious person got a loan, the company loses money. This loss is defined as:
\[ Rcost_N =   C_n * |N| \]

So, the monetary loss in case if a fraud detection model is not deployed can be calculated as follows:
\[ Loss' =  Rcost_N  +   Mcost_N  \]
\[ Loss' =  (2.100 * 743) + (0.5 * 743) \]
\[ Loss' =  1.560.671.5\]

The next step is to determine the possible impact of our best fraud detection model (PUL Ensemble without PCA and with custom voting, see the Figure~\ref{tables:confusion-matrices-ensemble} for the detailed performance result) on the calculated costs. 

The possible loss during the wrong classification of trustworthy applicants as fraudsters (missed profit here is the interest on loans) is calculated with the number of false positives:

\[ FNcost = r * |FN| * C_p = 0\]

because \[|FN| = 0.\]

Finally the gain caused by the correct classification of fraud attempts is given by:

\[Gain = C_n * |TN| = 2.100 * 542 = 1.138.200\]

The losses incurred due to wrongly issued loans to fraudsters and the marketing costs are \[2.100 * 201 + 0.5 * 201 = 422.200.5\]. Coupled with FNcost, the total loss amounts to $422.200.5$.

Thus as gains largely exceed losses when the PUL ensemble is applied, this justifies the business value of the proposed model for a given test data.

The calculations in this chapter are only preliminary and should be considered as a starting point. In the future, other business related factors, such as Customer Lifetime Value (CLF), need to be taken into account.






