\documentclass[draft=false
              ,paper=a4
              ,twoside=false
              ,fontsize=11pt
              ,headsepline
              ,BCOR10mm
              ,DIV11
              ]{scrbook}
\usepackage[ngerman,english]{babel}
%% see http://www.tex.ac.uk/cgi-bin/texfaq2html?label=uselmfonts
\usepackage[T1]{fontenc}
\usepackage[utf8]{inputenc}
%\usepackage[latin1]{inputenc}
\usepackage{libertine}
\usepackage{pifont}
\usepackage{microtype}
\usepackage{textcomp}
\usepackage[german,refpage]{nomencl}
\usepackage{setspace}
\usepackage{makeidx}
\usepackage{listings}
\usepackage{natbib}
\usepackage[ngerman,colorlinks=true]{hyperref}
\usepackage{soul}
\usepackage[printer]{hawstyle}
\usepackage{lipsum} %% for sample text
\usepackage{epigraph} %% for quotes



% custom
\usepackage{algorithm}
\usepackage[noend]{algpseudocode}
\usepackage{multicol}
\usepackage{color}
\usepackage{comment}
\usepackage{tikz}
\usetikzlibrary{positioning}





% LATEX DRAW REQ PACKAGES

 \usepackage[usenames,dvipsnames]{pstricks}
 \usepackage{epsfig}
 \usepackage{pst-grad} % For gradients
 \usepackage{pst-plot} % For axes
 \usepackage[space]{grffile} % For spaces in paths
 \usepackage{etoolbox} % For spaces in paths
 \makeatletter % For spaces in paths







%% define some colors
\colorlet{BackgroundColor}{gray!20}
\colorlet{KeywordColor}{blue}
\colorlet{CommentColor}{black!60}
%% for tables
\colorlet{HeadColor}{gray!60}
\colorlet{Color1}{blue!10}
\colorlet{Color2}{white}

%% configure colors
\HAWifprinter{
  \colorlet{BackgroundColor}{gray!20}
  \colorlet{KeywordColor}{black}
  \colorlet{CommentColor}{gray}
  % for tables
  \colorlet{HeadColor}{gray!60}
  \colorlet{Color1}{gray!40}
  \colorlet{Color2}{white}
}{}
\lstset{%
  numbers=left,
  numberstyle=\tiny,
  stepnumber=1,
  numbersep=5pt,
  basicstyle=\ttfamily\small,
  keywordstyle=\color{KeywordColor}\bfseries,
  identifierstyle=\color{black},
  commentstyle=\color{CommentColor},
  backgroundcolor=\color{BackgroundColor},
  captionpos=b,
  fontadjust=true
}
\lstset{escapeinside={(*@}{@*)}, % used to enter latex code inside listings
        morekeywords={uint32_t, int32_t}
}
\ifpdfoutput{
  \hypersetup{bookmarksopen=false,bookmarksnumbered,linktocpage}
}{}

%% more fancy C++
\DeclareRobustCommand{\cxx}{C\raisebox{0.25ex}{{\scriptsize +\kern-0.25ex +}}}

\clubpenalty=10000
\widowpenalty=10000
\displaywidowpenalty=10000

% unknown hyphenations
\hyphenation{
}

%% recalculate text area
\typearea[current]{last}

\makeindex
\makenomenclature

\begin{document}
\selectlanguage{english}


%%%%%
%% customize (see readme.pdf for supported values)
\HAWThesisProperties{Author={Ivan Morozov}
                    ,Title={Anomaly Detection in Financial Data by Using Machine Learning Methods}
                    ,EnglishTitle={Anomaly Detection in Financial Data by using Machine Learning Methods}
                    ,ThesisType={Masterarbeit}
                    ,ExaminationType={Masterprüfung}
                    ,DegreeProgramme={Master of Science Angewandte Informatik}
                    ,ThesisExperts={Prof. Dr. Kai von Luck \and Prof. Dr. Klaus-Peter Schoeneberg}
                    ,ReleaseDate={9. Mai 2016}
                  }

%% title
\frontmatter

%% output title page
\maketitle

\onehalfspacing

%% add abstract pages
%% note: this is one command on multiple lines
\HAWAbstractPage
{
    Maschinelles Lernen, 
    Betrugserkennung, 
    Finanzdaten, 
    Datenverarbeitung, 
    Support Vector Machine, 
    CRISP-DM, 
    Hauptkomponentenanalyse, 
    Ensemblemethoden, 
    Verhaltensdaten
}
{ 
 Die vergabe von Sofort-Krediten online ist eine moderne Business-Lösung. Ein Algorithmus basierend auf der Theorie vom maschinellen lernen entscheidet, ob ein Kredit vergeben wird oder nicht. Es ist nicht unüblich, dass Personen mit betrügerischen Absichten versuchen, das System zu umgehen - mit dem Ziel, einen Kredit zu bekommen. Diese Arbeit hat das Ziel, Methoden des maschinellen Lernens zu nutzen, um mögliche Betrugsfälle prognostizieren zu können. Die dafür verwendeten Daten werden während des Kreditantragsverfahrens gesammelt. 

Betrugs- bzw. anomale Fälle sind selten, so verwenden die Methoden in dieser Arbeit nur \textit{positive} (Kunden die Kredite zurückzahlen) und \textit{unmarkierte} (Kunden mit einem unbekannten Status der Rückzahlung) Daten um Betrugsfälle zu identifizieren. 

Eine Analyse der zugrunde liegenden Daten wurde durchführt und diverse Merkmale und Probleme wie z. B. die hohe Anzahl von fehlenden Daten wird diskutiert.

Drei Maschinen-Lern-Algorithmen werden vorgestellt. Die \textit{one-class SVM} verwendet nur positive Instanzen im Gegensatz zu \textit{Positive and Unlabeled Learning (PUL)} und \textit{PUL Ensemble}, wo beide - sowohl positive als auch unmarkierte - Daten verwendet werden.

Ein Experiment unter der Verwendung von Vorverarbeitungsoperationen und der diskutierten Algorithmen wurde durchgeführt. Dieser zeigte eine viel versprechende Betrugserkennungsrate bei der Verwendung von one-class SVM auf Kosten einer großen Anzahl von vertrauenswürdigen Bewerbern, die als Betrüger klassifiziert wurden. PUL verringerte die Anzahl der falsch vorhergesagten Rückzahler, während ein PUL Ensemble alle vertrauenswürdigen Bewerber richtig klassifizierte und die Betrugserkennungsrate auf etwa \(73 \% \) brachte. Zusätzlich wurde ein positiver Effekt der Vorverarbeitung von Daten und eine negative Auswirkung der Hauptkomponentenanalyse entdeckt. Schließlich wurde der mögliche Effekt von PUL-Ensemble als ein Teil des Kredit-Scoring-Systems berechnet.

Die Arbeit kommt zu dem Schluss, dass eine erfolgreiche Betrugserkennung auch ohne die Verwendung bereits dokumentierter Betrugsfälle möglich ist.
}
{
    Machine Learning, 
    Fraud Detection, 
    Financial Data, 
    Big Data, 
    Support Vector Machine, 
    CRISP-DM, 
    One Class Support Vector Machine, 
    Principal Component Analysis, 
    Ensemble Methods, Behavior Data, 
    Positive and Unlabeled Learning
    
}
{ 
      The instant online issuing of micro-loans is a modern credit lending business solution. It is based on a machine learning algorithm that automatically scores loan applications.
      It is not uncommon that some malicious persons try to bypass the system and get a loan. 
      This thesis aims to utilize advanced machine learning methods to predict possible fraud on data collected during the credit-application process. 

      Fraudulent/anomalous cases are by definition rare, thus, the machine learning methods discussed in this thesis are based on discriminating fraudsters by using only positive (good customers that repay loans) and unlabeled (customers with a still unknown repayment status) data. 

      A brief analysis of the underlying data is performed and several characteristics and issues like the high amount of missing values are discussed.

      Three machine learning algorithms able to learn only from positive and unlabeled data are introduced. The one-class Support Vector Machine (SVM) uses only positive instances in contrast to Positive and Unlabeled Learning (PUL) and PUL Ensemble, where both - positive and unlabeled data are used. 

      An experiment that utilizes the preprocessing operations and the machine learning algorithms showed that a one-class SVM can deliver a promising fraud detection rate but at the expense of a large number of trustworthy applicants being misclassified; PUL significantly decreases the false negative rate but failed to detect more fraud cases, whereas a PUL ensemble achieves a zero false negative rate while driving the fraud detection rate to about \(73\%\). In addition, a positive effect of the preprocessing pipeline and a negative impact of Principal Component Analysis is discovered. Finally, a business value of deploying a PUL ensemble as a part of credit scoring system is calculated for a given test set.

      The thesis came to the conclusion that a successful fraud detection is possible even when fraud cases are rare or unavailable.
}

\newpage
\singlespacing
\section*{Acknowledgement}
At this point I would like to acknowledge all the people who supported and assisted me through the work on this thesis.

First, I would like to thank my advisors Prof. Dr. Kai von Luck and Prof. Dr. Klaus-Peter Schoeneberg at HAW-Hamburg. Prof. von Luck was always available to answer any questions I had and has played an important part in keeping me motivated. I am also very thankful for all the valuable comments and recommendations provided by Prof. Schoeneberg. The expertise of Prof. von Luck and Prof. Schoeneberg has left a huge impact on me and on my work. 

I also want to express my deepest gratitude to Dr. Oleg Okun who was one of my advisors at Kreditech. His patience, support and valuable advice have helped me move in the right direction. His ability to explain complex things in a simple way is unique and made my work a lot easier.

Additional gratitude goes to Dr. Liuben Siarov who was also an advisor to me at Kreditech. He guided me in my research, helped me structure my work and motivated me to approach advanced topics.

Further acknowledgement and thanks are due to Alexander Graubner-Müller, CEO at Kreditech, who was the one who suggested the idea of anomaly detection and created the environment to work on it in his company.

Finally, I want to thank all my colleagues at Kreditech for their support. They have made the writing of this thesis in a real-world company as convenient as possible by answering all my questions and providing me with information and tools necessary to complete my work.



\tableofcontents
\newpage
%% enable if these lists should be shown on their own page
\listoftables
\listoffigures
%%\lstlistoflistings

%% main
\mainmatter
\onehalfspacing
%% write to the log/stdout
\typeout{===== File: chapter 1}
%% include chapter file (chapter1.tex)

\chapter{Introduction}

The Concise Oxford Dictionary of Mathematics defines an anomaly as an unusual and possibly erroneous observation that does not follow the general pattern of a drawn population \cite{Clapham:2013}. Anomaly detection is a branch of data mining that seeks to find data points or patterns that do not fit the overall pattern of the data. Studying anomalous behavior has been applied in many branches of science. For example, an anomalous pattern in financial transaction data could be indicative of fraudulent activity or a an anomalous MRI image may indicate the presence of malignant tumors \cite{Spence:2001:DSC:882464.882797}. Anomaly detection's practicability relies heavily on the simple statistical assumption that rare observations may carry critical information \cite{Chandola:2009:ADS:1541880.1541882}. Detection of an anomaly can have a measurable positive effect and bring far reaching benefits. In the following work I will present anomaly detection using data mining methods on a data set containing credit application forms. The application forms have been submitted via a web interface and include mostly rudimentary personal- as well as financial-information. To collect more information about the credit applicant, the system also tracked technical metrics such as used web-browser and screen resolution. A detailed description of input data will be introduced in the second chapter of this thesis.


\section{Types of Anomalies}

In the context of anomaly detection, anomalies are divided into three categories:

\begin{itemize}
    \item \textit{Point} is the simplest type of an anomaly, it describes an outlier of a single data instance with respect to other data \cite{Chandola:2009:ADS:1541880.1541882}. Given a measurement in several dimensions the point anomaly only affect one of the dimensions. A real world example in financial context can be as follows: Our system predicts anomalies in credit applicant data, one of the data instances is the applicants salary. A salary that is extreme high with respect to the rest of data is a point anomaly.

    \item \textit{Context} is a extension of a point anomaly. A context anomaly describes a possible outlier with respect to it's context. Given a measurement in several dimensions we find that one dimension is anomalous with respect to a different dimension of the same measure. In context anomalies a outlier can only appear in the context of all involved data points and only then \cite{Chandola:2009:ADS:1541880.1541882}. A example from real world can be a credit applicant system described above. Let us assume that we have one one applications with an extreme high salary, if we predict solely based on salary then the application may be indicated anomalous. However, adding the country as another data instance, our model could detect that with respect to the context of country the given salary is not anomalous.

    \item \textit{Collective} anomalies can be detected by observing a collection of relative data points in a data set. Individual data points from the observed collection or sub sequences my not anomalies by themselves, but occurred together they are anomalous \cite{Chandola:2009:ADS:1541880.1541882}. Imagine a credit applicant with following data instances: salary, country, amount applied for. For the sake of simplicity we assume that the credit amount applied for is lower then the salary. Such low application amount is not an anomaly by himself, but in composition with the salary and the country it could be identified as an anomalous pattern for fraudulent behavior. This kind of anomalies have been explored for sequential data \cite{Warrender99detectingintrusion}. This thesis will only treat anomalies in non-sequential data and thus collective anomalies will not be discussed any further.
    
\end{itemize}

\section{Challenges}
Determining whether an observation is an anomaly or a normal observation is closely tied to the questions: what is the general pattern of the data and what is normal according to that pattern? Finding the general pattern of data is subject of active research in statistics, machine learning and artificial intelligence. Drawing a clear line between these fields is irrelevant as many practical methods are a composition rather than a disjunction. This Thesis will treat a machine learning centered approach. A straight forward approach to determine the general pattern of underlying data is creating and training a prediction model. Taking this approach I will have to deal with the following challenges.


\begin{itemize}
    \item Availability of data which label significant anomalous patterns is a common problem in anomaly detection topic. Rare cases are by definition less common in sample data \cite{Weiss:2004:MRU:1007730.1007734}. Identifying them by hand requires domain knowledge and sometimes impossible due to the fact that they haven't occurred yet.
    
    \item Anomalous pattern are not static and can evolve over time. A pattern that is anomalous at a given point in time must not be anomalous in the future and vice versa \cite{Chandola:2009:ADS:1541880.1541882}.
    
    \item To establish the \textit{ground-truth} about anomalous data we need to be provided with sufficient data. Rarity makes result evaluation to a significant problem, in other words with non availability of labeled data, we do not have the base case to compare with \cite{Aggarwal:2013}. %seite 33
    
    \item Distinguishing between noise and anomalies in data leads to a high false positive rate \cite{Chandola:2009:ADS:1541880.1541882}.   % noise in glossary, false positive in glossary
    
\end{itemize}

\section{Machine Learning} 

Deciding on the right machine learning technique to detect anomalies is highly dependant on the model purpose as well as available data. The best choice of a model requires a deep domain knowledge and analysis of available data \cite{Aggarwal:2013}. The relevant data for the described analysis is high dimensional data that contains a small set of positive labeled examples. This observation leads to the fact that we should only consider \textit{semi-supervised} and \textit{unsupervised} learning methods. Moreover we should consider that anomalies in financial data can be a result of malicious actions that were intented to look like normal data. This indicates that the data is noisy and the application we use should have a stable strategy to deal with noisy data. 
In this chapter I will give a brief introduction in  machine learning methods used for anomaly detection and tradeoff their beneficing for our use case. The methods we decide for will be discussed in detail in the third chapter of this work.

\subsection{Extrema-value analysis}

Is an distribution based technique and is one of the most basic form of outlier detection in 1-dimensional data. The core point of this methodology is to find statistical tails with respect to underlying data distribution \cite{Aggarwal:2013}.

Naturally designed for univariate data, extreme value analysis not suit for our use case with high dimensional input data. Nonetheless we should consider that extreme-value analysis play an important role as a final step in most outlier detection algorithms where in final step the data set is represented as a set of univariate values.

\subsection{Proximity-based methods}

The major idea of proximity-based methods is to pool related data sets to groups or cluster based on available data. Data points that are not been allocated to any groups are mostly the outliers.

\textit{Density} based techniques like the nearest neighbour are vulnerable to noise in high dimensional data. Relative contrast between data points decrease with increasing dimensionality \cite{Hinneburg:2000:NNH:645926.671675}. Considering that our sample data is high dimensional this behavior can increase false-negative rate and is thus negligible. 

\begin{figure}
\centering
    \includegraphics[scale=0.4]{Graphics/plottA.png}
\caption{ X-Y Axis}
 \includegraphics[scale=0.4]{Graphics/plottb.png}
 \caption{ X-Z Axis }
\label{fig:clustering_dimensions}
\end{figure}

\textit{Clustering} algorithms such as the k-means do not work well in high-dimensional space. In  real life in many real data example points are correlating with respect to a given set of dimensions and others with respect to different dimensions \cite{Aggarwal:1999:FAP:304181.304188}. The figure \ref{fig:clustering_dimensions} shows an 3-dimensional space with two patterns, one in \(X\) - \(Y\) plane and another in \(X\) - \(Z\). Clustering in the full dimensional space will not discover the two patterns, since each of them is spread out along one of the dimensions \cite{Aggarwal:1999:FAP:304181.304188}.The weaknesses of proximity-based methods in respect to our use case are the high computation time \(O(2)\) and the fact that anomalies are often lost in the noise when methods are applied on data with a high factor of dimensionality \cite{Aggarwal:2013}.

\subsection{Classification}

The nature of anomalies in Financial data is often highly specific to appropriate kinds of abnormal activity \cite{Liu:2008:SMS:1425519.1425528,Aggarwal:2013}, in this case the underlying data contains noise. Supervised methods provide the ability to integrate domain specific knowledge in form of labeled data into detection process to obtain more meaningful anomalies. An general recommendation given by C.C Aggarwal in his book \textit{Outlier Analysis} is \textit{always use supervision where possible} \cite{Aggarwal:2013}. The challenge in supervised learning applied on anomaly detection is that labels are extremely unbalanced in terms of relative presence \cite{Chawla:2004:ESI:1007730.1007733}, this is called the \textit{rare case problem} \cite{Weiss:2004:MRU:1007730.1007734}.

Faced with the fact that the anomalous class examples are sparsely available but we still prefer to use supervision we have to label at least an subset of our underlying data. \textit{Positive Unlabeled Learning (PU Learning)} is an classification approach where the data is separated in two sets: \(P\) contained positive labeled examples and \(U\) that can contain outliers as well as positive unlabeled data. 

\textit{Support Vector Machines} (Cortes and Vapnik, 1995) is an state-of-the-art classification method with a solid theoretical background in quadratic optimisation. It introduce a method to find a hyperplane that splits initial data in two or more classes. One of the main advantages is the ability to separate non linear data by casting it into a higher dimensional space where the data is separable. 
\textit{One Class SVM} is an SVM based classifier method. In contrast to the conventional SVM it has knowledge about only one class inside the data sample. 
\textit{Ensamble SVM} is an approach to combine multiple SVM Algorithms. The general idea of sequential ensembles is to provide a better understanding of the data, so as to enable a more refined execution with either a modified algorithm or data set \cite{Aggarwal:2013} by applying the same or several classification algorithms on the same data and evaluation the results. Alternatively independent ensambles use different initialisations of same algorithm or different sub sets of underlying data. The results can be combined to obtain an more robust outlier score \cite{Aggarwal:2013}.

\subsection{Machine Learning Result Evaluation}

Evaluating the results of outlier detection algorithms and measuring their effectiveness is an essential task. The main requirement for evaluation is the availability of ground-truth about which points are outliers in data. Since the ground-truth is available an part can be used for training and the remaining for evaluation. 

\textit{Receiver operating characteristic curve (ROC Curve) } is a technique to visualising binary classifier performance through measuring the tradeoff between hit rates and false alarms \cite{Fawcett:2006:IRA:1159473.1159475}. Results of binary classifier like One Class SVM can be easily adapted to use ROC Curve for performance evaluation. % more here??

\section{Current State in Research}

Anomaly detection has been wide reviewed topic by the statistical community. The classic literature from the perspective of statistics \cite{Hawkins:1980,Barnett:1978} was written before the wider adoption of database technology, and are therefore not written from a computational point of view \cite{Aggarwal:2013}.
Increasing performance of data processing allowed studying the problem of Anomaly Detection more recently by the computer science community. Numerous scientific surveys \cite{Agyemang:2006:CSN:1609942.1609946,Chandola:2009:ADS:1541880.1541882,Chandola:2012:ADD:2197072.2197116,Pimentel:2014:RRN:2588908.2589196} discuss anomaly detection for particular domains from different points of view. Charu C. Aggrawal provide in his book \textit{Outlier Analysis} \cite{Aggarwal:2013} a comprehensive evaluation of outlier detection techniques from the data mining perspective. 

An general consistent statement in anomaly detection surveys is that the algorithm to use is highly dependant on the domain specific use case and the available data \cite{Agyemang:2006:CSN:1609942.1609946,Chandola:2009:ADS:1541880.1541882,Chandola:2012:ADD:2197072.2197116,Pimentel:2014:RRN:2588908.2589196}. Table \ref{tab:recentwork} illustrate previous works in topic of anomaly detection that will be helpful for further research in respect to our use case of outlier detection in financial data.

\begin{table}
  \begin{center}
    \caption{Recent work review}
    \label{tab:recentwork}
    \begin{tabular}{l|l}
    Technique & References \\
      \hline
     SVM & \cite{Eskin:2010,Chandola:2009:ADS:1541880.1541882,Hinneburg:2000:NNH:645926.671675,Aggarwal:1999:FAP:304181.304188,Ahmed:2015,Claesen:2014} \\
     \hline
     Proximity Based & \cite{Chandola:2009:ADS:1541880.1541882,Hinneburg:2000:NNH:645926.671675,Aggarwal:1999:FAP:304181.304188,Ahmed:2015} \\
     \hline
     PU Learning & \cite{Li:2011, Claesen:2014} \\
     \hline
     Ensamble methods & \cite{Peddabachigari:2007, Bukhtoyarov:2014, Kavitha:2015, Claesen:2014} \\
     \hline
    \end{tabular}
  \end{center}
\end{table}

The reviewed publications does not contain examples with case of anomaly detection in credit applicant data. However the topic of credit card fraud \cite{Eskin:2010,Chandola:2009:ADS:1541880.1541882,Hinneburg:2000:NNH:645926.671675,Aggarwal:1999:FAP:304181.304188,Ahmed:2015} and network intrusion \cite{Peddabachigari:2007,Kavitha:2015,Eskin:2010,Chandola:2009:ADS:1541880.1541882,Hinneburg:2000:NNH:645926.671675,Aggarwal:1999:FAP:304181.304188,Ahmed:2015} are well researched. The insights can be combined and used to develop an own outlier detection model.

For the sake of content completeness table \ref{tab:recentworkbase} referring previous works in theoretical field of machine learning describing techniques i will use.

\begin{table}[h!]
  \begin{center}
    \caption{Recent work review}
    \label{tab:recentworkbase}
    \begin{tabular}{l|r}
    Technique & References \\
      \hline
     SVM & \cite{Tax:2004:SVD:960091.960109} \\
     \hline
     ROC & \cite{Fawcett:2006:IRA:1159473.1159475} \\
     \hline
     Ensembles & \cite{Nikunj:2009,Caruana:2004:ESL:1015330.1015432} \\
     \hline
    \end{tabular}
  \end{center}
\end{table}



\section{Thesis Goal}
The goal in following work is to evaluate several machine learning algorithms for anomaly detection in real world business use case. 
The goal in following work is to detect anomalous patterns in financial data which can be indicative for fraudulent behaviour. 

\section{Thesis Structure}

This thesis will be structured as follow. In second part we will care about our input data instructed by the CRISP-DM process. The subsection acquisition will help to understand the data from the real world point of view. In the exploration part we will discuss the quality and expressiveness of the input data. The last subsection of second chapter will contain the prepossessing part that is necessary to use it with machine learning methods. The third chapter will contain a theoretical overview about Machine Learning Algorithms we apply on our data followed by a chapter where we compare the methods to each other and evaluate the advantages and disadvantages in respect to our use case. The fifth chapter is the experimental part, it will contain the setup and results of applying machine learning methods on preprocessed data. In following chapter we discuss the results of the experiments, a core topic of the discussion will be the possible business impact of using our model in financial use case. The last chapter contains a summary about the work that has been done and will end with a overview of perspectives for the future research in anomaly detection topic.
\chapter{Data Processing}



\section{Acqusition}


\subsection{Overview (business understanding)}



\subsection{Feature Description}



\section{Exploration}



\subsection{Statistical Summary of Data}



\subsection{Visual Summary of Data} /* Histogramms Plotts and so on */




\subsection{Correlation Analysis} /* Correlation Plot and description */


\subsection{Data Quality (missing values)} /* Missing values where? */




\section{Preprocessing}
\subsection{Missing Value Imputation} /* Missing values impute */
\subsection{Categorical to Numeric Transformation }
\subsection{Removing correlated Features } 
Correlations are noise and increase computation time
\subsection{Removing linear Combinations of Features} /* Missing values where? */
\subsection{Feature Normalization} /* Missing values where? */


\chapter{Machine Learning Methods}\label{ch:3}
This chapter will treat the machine learning methods applied in this thesis. The subsection: \ref{Chapter:PUL} will discuss the Positive Unlabeled Learning (PUL) strategy for \textit{training} of our prediction model, in the following: \ref{Chapter:SVM} section, this thesis will point out the  idea of Support Vector Machines (SVM) including an introduction into the theoretical parts. The subjecting chapter: \ref{Chapter:OC-SVM} approach on One Class SVM, it will cover the concept and the gaps in respect to the ordinary SVM. At the end of this chapter, the subsection: \ref{Chapter:Ensemble} will introduce the concept of Ensambles in scope of machine learning especially to improve the classification accuracy.

The purpose of the following introduction is:
\begin{itemize}
    \item Understand the core concept of the particular algorithms.
    
    \item Get a brief understanding of the theoretical concepts driving the algorithms.
    
    \item Became familiar with the possible tuning options that the algorithms contain and be able to track the motivation of adjusting them.
\end{itemize}
One last note before starting: the theoretical concepts will be presented in an aggregated form. The goal is to provide only the information necessary to follow up in the further chapter. 

\section{Positive Unlabeled Learning}\label{Chapter:PUL}
Faced with the fact that the anomalous class examples are sparsely available but we still prefer to use supervision we have to label at least an subset of our underlying data. \textit{Positive Unlabeled Learning (PU Learning)} is an classification approach where the data is separated in two sets: \(P\) contained positive in our case examples that are labeled as non anomalous and \(U\) that can contain outliers as well as positive unlabeled data.

\section {Support Vector Machines (SVM)}\label{Chapter:SVM}

%http://stats.stackexchange.com/questions/63028/one-class-svm-vs-exemplar-svm
Support Vector Machines (SVM) \cite{Cortes;Vapnik:1995} is a state-of-the-art method with a solid background in statistical learning theory. It can be used for both classification and regression tasks. What distinguishes an SVM from other methods is a better ability to deal with high dimensional data and the guarantee of the globally optimal solution. The solution an SVM produces is sparse in many cases as only a fraction of training set instances is relevant for the task at hand. These instances, called support vectors, lie close to a hyperplane separating data into classes. Thus, an SVM tries to transform nonlinearly separable classes into linearly separable ones because the latter case is simpler to solve than the former. Without loss of generality and for the purpose of this thesis, only one or two classes are assumed to be present in the data.

Let us assume that we are given a data set as
\[  S = \{(x_1,y_2), (x_2,y_2), ...,(x_n,y_n) \} \; x_i \in \mathbf{R^d} \; y_i \in \{-1,1\}, \]
where \( x_i\) is the \( i-\)th input instance or data point and \( y_i\) is its class label. Thus, \(x_i\) is a \(d-\) dimensional column vector whereas \( y_i\) is a scalar.

A hyperplane that splits the data into two classes can be represented with the following equation:
\[\vec{w}^Tx + b = 0, \] 
where \(\vec{w}\) is a \textit{weight vector} determining the direction perpendicular to the hyperplane and \(b\) a \textit{bias} responsible for moving the hyperplane parallel to itself (see also~\label{fig:hyperplane1}.

\begin{figure}[h!]
    \centering % Picture from Oleg's book, do not forget the reffs!
    \includegraphics[scale=0.4]{Graphics/svm1.png}
    \caption{Hyperplane for separating 2-dimensional data.}
    \label{fig:hyperplane1}
\end{figure}

However, classes in the input space are often not linearly separable, which means that a linear classifier is not a good option in such a case. In the case of SVMs a solution is to project the original data into another, often a higher dimensional space \(x \mapsto \phi(x) \), where classes would more likely be linearly separable. Figure~\ref{fig:hyperplane2} shows an example of input space \(X\) where data cannot be separated by a linear function. However after applying the mapping function \(\phi\) to each data point in \(X\), the data become well separable in a \textit{feature space} \(F = \{ \phi(x)\; | \; x \in X\}\).
\begin{figure}[h!]
    \centering
    \includegraphics[scale=0.6]{Graphics/svp-separation.png}
    \caption{Mapping data into feature space.}
    \label{fig:hyperplane2}
\end{figure}

Thus, a straightforward solution seems to transforms data into a feature space where a linear classifier can be built. These two operations are combined with the help of a kernel function. The typical kernel functions are:
\begin{itemize}
            \item \(K(x,z) = x'z \)
            \item \(K(x,z) = (\tau + x'z)^b\)
            \item \(K(x,z) = exp(-\sigma||x-z||^2)\)
\end{itemize}

As one can see, the kernel representation eliminates the necessity to map each input individually: the inputs never appear isolated but in the form of inner products between pairs of inputs. Because of this, we don't need to know the underlying feature map! Also the dimensionality of the feature space does not affect the computation as the inner product is a number. As a result, the only information that is necessary is a \(n\times n\) kernel matrix.

Kernels provide one pillar of SVMs. The other is the optimization theory as the SVM solution is formulated as an optimization task, subject to certain constraints. The primal optimization problem where \( w \) and \( b \) are involved is difficult to solve due to inequality constraints. Instead the dual problem, based on  Lagrangian theory \footnote{Lagrangian theory is a basic mathematical tool for constrained optimization of differentiable functions, especially for nonlinear constrained optimization \cite{Li:2008}.} transforms the task into a quadratic programme where the function to be optimized is quadratic while the constraints are all equalities rather than inequalities. The solution of such a problem is known to be unique and global. It is also sparse by implying that only a small fraction of the original data matters for class separation, which results in a very efficient classifier.

Below both primal and dual optimization problems are given. The maximal (or hard) margin problem assumes two classes are only linearly separable in the feature space. To remedy its deficiency, the soft margin problem are then presented that works with nonlinearly separable classes.

The maximal margin:

Primal problem:

\[\textrm{minimize: } \vec{w} \vec{w},\] 
\[\textrm{subjects to: } y_i(\vec{w}\vec{x_i} + b) \geq +1, i = 1, ...,l \] 

Dual problem:
\[\textrm{\textit{maximize: } } W(a) =  \sum_{i=1}^{l} a_i \; - \frac{1}{2}\sum_{i,j = 1}^{l}a_i a_j y_i y_j \vec{x_i}^T \vec{x_j}, \] 

\[\textrm{subjects to: } \sum_{i=1}^{l}a_i y_i = 0, a_i \geq 0, i = 1,...,l.   \] 

The 2-norm soft margin:

Primal problem:

\[\textrm{minimize: } \vec{w} \vec{w} + C \sum_{i=1}^{l} \xi^2_i \textrm{over} \xi,\vec{w},b\] 
\[\textrm{subjects to: } y_i(\vec{w}\vec{x_i} + b) \geq 1 - \xi_i,, i = 1, ...,l \] 

Dual problem:
\[\textrm{maximize:} W(a) =  \sum_{i=1}^{l} a_i \; - \frac{1}{2}\sum_{i,j = 1}^{l}a_i a_j y_i y_j \Big( K\vec{x_i}^T \vec{x_j}) + \frac{1}{c}\delta_i_j \Big), \] 

\[\textrm{\textit{subjects to: }} \sum_{i=1}^{l}a_i y_i = 0, a_i \geq 0, i = 1,...,l.   \] 


The 1-norm soft margin:

Primal problem:
\[\textrm{minimize: } \vec{w} \vec{w} + C \sum_{i=1}^{l} \xi_i \textrm{over} \xi,\vec{w},b\] 
\[\textrm{subjects to: } y_i(\vec{w}\vec{x_i} + b) \geq 1 - \xi_i,\xi_i \geq 0, i = 1, ...,l. \] 

Dual problem:

\[\textrm{maximize: } W(a) =  \sum_{i=1}^{l} a_i \; - \frac{1}{2}\sum_{i,j = 1}^{l}a_i a_j y_i y_j (\vec{x_i}^T \vec{x_j}) , \]
\[\textrm{subjects to: } \sum_{i=1}^{l}a_i y_i = 0,C \geq a_i \geq 0, i = 1,...,l.   \] 

For the hard margin classifier SVM introduce additional classification hyperplanes:
\[H_1 : \vec{w}^Tx_i^- + b= -1\] 
\[H_2 : \vec{w}^Tx_i^+ + b= +1\] 
each of them cross at least the points of the particular classes which are the nearest points in respect to the separating hyperplane.
\begin{figure}[h]
    \centering
    \includegraphics[scale=0.4]{Graphics/svm-margins.png}
    \caption{Separator hyperplane and two margin planes.}
    \label{fig:hyperplane3}
\end{figure}
The goal is to find the widest possible margin between the separator hyperplane \(H_0 \longleftrightarrow H_i\) and the classification hyperplanes. The margin is defined as an optimization problem:
\[ \textrm{\textit{maximize} } \frac{2}{||\vec{w}||} \mapsto \textrm{\textit{minimize} } \frac{||\vec{w}||}{2} \mapsto \textrm{\textit{minimize} } \frac{1}{2}||\vec{w}||^2\]
\[\textrm{With constraints:} \]
\[  y_1 = +1, \; \vec{w}^Tx_1 + b = \; \geq +1\]
\[  y_2 = -1, \; \vec{w}^Tx_2 + b = \; \leq -1\]
\[\textrm{Multiplying the constraints with each labels so both     constraints can be substituted to a single one:}\]
\[ y_i(\vec{w}^Tx_i + b) \geq +1\]
Solving constrained problems by using Lagrangian multipliers. The \textit{primal} optimization problem is defined as: 
\[\textrm{\textit{minimize} } L_p(\vec{w},a_i,b) = \; \frac{1}{2}||\vec{w}||^2 - \; \sum_{i=1}^{n} a_i(y_i(\vec{w}^T x_i +b) - 1) \]
\[\textrm{such that } a \geq 0 \]
Subjecting the primal representation to the so called \textit{dual} representation which often turns to be easier to solve than the primal problem since handling inequality constraints directly is difficult.
%here link to Oleg's book.
\[\textrm{\textit{maximize} }  \sum_{i=1}^{n} a_i \; - \frac{1}{2}\sum_{i=1}^{n}\sum_{j=1}^{n}a_i a_j y_i y_j \vec{x_i}^T \vec{x_j} \]
\[\textrm{such that } a \geq 0, \textrm{ and } \sum_{i=1}^{n}a_i y_i = 0  \]
To solve the optimization problem we need to compute the dot product between all pairs of \(\vec{x_i}^T \vec{x_j}\). After solving the optimization problem we are able to define the decision rule to classify new data:
\[  f(x) = \vec{w}^T x + b = \sum_{i=1}^{n}a_i y_i x_i^T x + b \; \geq 0  \]
Where \(x_i\) are the support vectors and \(x\) is the test point. \newline

A quick summary of things we already discussed:
\begin{itemize}
    \item SVM classify the data by finding a optimal linear hyperplane which split the data in two or more classes.
    
    \item The algorithm of SVM can linear separate non linear data by casting it into a higher space with the kernel method.
    
    \item The core idea of SVM is to find a widest possible margin between separator hyperplane and the support hyperplanes. Bigger the margin better classification.
    
    \item Computing the biggest margin subjects to an optimization task in quadratic programming. The computation process can also be called \textit{training} in machine learning context.
\end{itemize}

As last but not least this chapter will introduce an important concept to optimize the results produces by SVM, called \textit{soft margin optimization}.

Most real world data consist \textit{noise} caused by overfitting (high complexity of input data), this can result in poor generalisation and negative affect the performance of SVM. The concept of soft margin optimization can be viewed as the relaxation of the hard margin constraint discussed above. First of all we introduce two \textit{slack} variables: \(\xi \) which allow some data to overlap the separation boundary see figure: \ref{fig:slackvariable} and \(C > 0\)  a positive scalar as the regularization parameter to control possible overfitting.
\begin{figure}[h!]
    \centering
    \includegraphics[scale=0.4]{Graphics/svm-slack-variable.png}
    \caption{Slack variable example.}
    \label{fig:slackvariable}
\end{figure}
Adding the slack variables to our optimization problem we get:

\[ min_{w,b,\xi} \;\; \frac{1}{2}w^Tw+C \sum_{i=1}^{l} \xi_i\]
\[ \textrm{such that } y_i(w^T\phi(x_i) + b) \geq 1 - \xi_i, \]
\[\xi_i \geq 0, i = 1,...,l \]

A final summary of this chapter:

\begin{itemize}
    \item SVM introduce a concept by dealing with noise called soft margin optimization.
    
    \item Soft margin optimization use slack variable \(\xi \) to allow data points overlap the separation boundary and \(C\) as upper bound to prevent overfitting. Both variables are the typical \textit{tuning parameter} in the common usage of SVM.
\end{itemize}


\subsection{One Class SVM}\label{Chapter:OC-SVM}
\textit{One Class SVM} is an SVM based classifier method. In contrast to the conventional SVM it has knowledge about only one class inside the data sample. 

\section{Ensemble SVM}\label{Chapter:Ensemble}
\textit{Ensamble SVM} is an approach to combine multiple SVM Algorithms. The general idea of sequential ensembles is to provide a better understanding of the data, so as to enable a more refined execution with either a modified algorithm or data set \cite{Aggarwal:2013} by applying the same or several classification algorithms on the same data and evaluating the results. Alternatively independent ensambles use different initialisations of same algorithm or different sub sets of underlying data. The results can be combined to obtain an more robust outlier score \cite{Aggarwal:2013}.

%http://www.outlier-analytics.org/odd13kdd/papers/slides_charu_aggarwal.p



% FAZIIIIIIT!!!!!



\chapter{Performance evaluation of Machine Learning Methods}


\chapter{Experimental Protocol}\label{Chapter:5}

The following chapter the design aspects introduced in previous chapters (\ref{ch:2},\ref{ch:3},\ref{Chapter:4}) are brought together and find their practical application. The goal is to produce measurable performance results  by applying of  preprocessing methods and statistical algorithms.

The general pattern that the experiment chapter is followed is partially inclined to the CRISP-DM process (see figure: \ref{fig:crisp-dm}) and contain three sub-processes: \textit{DataPreparation, Modelling, Evaluation}. Each of them contains techniques previously described in this Thesis. 

\begin{figure}[h!]
    \centering
    \includegraphics[scale=0.5]{Graphics/DeploymentDiagram1.png}
    \caption{General experiment pattern} %footnote?
    \label{fig:dep-dia}
\end{figure}

An experiment is defined as an process that follow the general experiment pattern (see figure: \ref{fig:dep-dia}), where the particular methods (see tables: \ref{tab:data-pred-methods}, \ref{tab:machine-learn-tech}, \ref{tab:res-ev}) and their parameters can vary.

This chapter is structured in two sections: First, the initial techniques and settings are given in chapter \ref{ch:5:ES}, then the results are presented in chapter \ref{Chapter:5:Results}. 


\section{Experiment Settintg (plan)} \label{ch:5:ES}

\textbf{Dataset description:}

The initial data set has \(95.951\) instances on \(1804\) features\footnote{A detailed overview of underlying data can be found in Chapter \ref{Ch:2:FeatureDesc}}. 
For all experiments, that are using algorithms based on positive and unlabeled data (One-Class SVM, PUL, SVM Ensemble), the data is divided into training, validation and test sets \((60\%/30\%/10\%)\).  All instances that are labeled as fraud \((744)\) are a member of the test set since the algorithms are utilized to treat unlabeled and  positively labeled data. 

All classification tasks are using 10-fold cross-validation. 


\begin{table}[ht!]
\centering
\setlength\tabcolsep{4pt}
\begin{minipage}{0.30\textwidth}
\centering
%\tablewidth=\textwidth
 \begin{tabular}{|l|l|}\hline
  Name&Reference\\ \hline
 Removing Corrupted Examples  & \ref{Ch:2:RCD} \\ \hline
        Categorical to Numeric Transform. & \ref{Ch:2:CTNT} \\
     \hline 
      Missing Value Imputation & \ref{Ch:2:MVI} \\
     \hline 
     
     Removing Zero- and Near-Zero Variance Features & \ref{Ch:2:RNZVF} \\
     \hline 

     Principle Component Analysis (PCA) & \ref{Ch:2:PCA} \\
     \hline 
        \end{tabular}
\caption{Data Preparation Methods.}
\label{tab:data-pred-methods} 
\end{minipage}%
\hfill
\begin{minipage}{0.24\textwidth}
\centering
 \begin{tabular}{|l|l|}\hline
         Name & Reference\\
        \hline
        Two-Class SVM & \ref{Chapter:SVM} \\
        \hline
        One-Class SVM & \ref{Chapter:OC-SVM} \\
        \hline
        PU-Learning &  \ref{Chapter:PUL} \\
        \hline
        Ensemble SVM &  \ref{Chapter:Ensemble} \\
        \hline
        
    \end{tabular}
 \caption{Machine Learning Techniques.} 
 \label{tab:machine-learn-tech} 
\end{minipage}
\end{table}

\begin{table}[ht!]
    \begin{center}
    \caption{Machine Learning Results Evaluation Techniques.}
    \label{tab:res-ev}
        \begin{tabular}{|c|c|}
        \hline
        Name&Reference\\
        \hline
        ROC-Analysis & \ref{Chapter:4:ROC} \\
        \hline
        \end{tabular}
\end{center}
\end{table}



\textbf{Preprocessing:}

Here, the prepossessing part will pass all steps given in Table ~\ref{tab:data-pred-methods}. The imputation of missing values is done by imputing the \textit{mean} value for numerical and by adding of a constant to categorical features. Removing low variance features go with a \(freqCut = 95/5\) (the cutoff for the ratio of the most common value to the second most common value).
The PCA analysis should capture \(95\%\) of the variance.

\textbf{Machine Learning:}

There are three different models (see Table ~\ref{tab:machine-learn-tech} in this experiment that are utilized for the case of detection of anomalous data which are indicative for fraud cases.

The one-class SVM is trained with Gaussian Radial Basis kernel, the hyper-parameter \(\sigma\) is estimated by the \textit{sigest} heuristic\footnote{An heuristic provided by the \textit{kernlab} library used to compute the one-class SVM \cite{Kernlab}}, cost of constraints violation is set to \(C = 1\) and slack variables to \(\xi = 0.2\).

PUL requires an estimation of the conditional probability of an example has been labeled. Therefore, a two-class SVM with \textit{Weighted RBF-Kernel}\footnote{Since the class membership inside the training set is imbalanced (see Table ~\ref{tab:instance-summary}), the weighted RBF kernel is able to prior weights (e.g give less present class more priority) and is thus more suitable for imbalanced data sets \cite{6351707}.}. The tuning parameters (\(\sigma\) (Sigma), \(C\) (Cost), \(Weight\) (Weight)) are not static set and will be guessed automatically by a heuristic search \footnote{The library \textit{caret}, used for computing the SVM provide a method where the parameters are guessed automatically by a heuristic search \cite{JSSv028i05}.}. The resulted classifier \(g(x)\) is then used to predict the probability of being labeled on the validation set. The mean probability of being labeled is then set as \(c\). As last step \(g(x)\) is applied to the test set, the probabilities of being labeled are divided by \(c\) and classified as an outlier if the result probability is less than 0.5 \cite{Li:2011}. 


Ensemble SVM will be trained with 10 random draws (for \(P\) and \(U\)) of training data. The CWSVM is a two-class SVM with the same parameters that are used by PUL. 




\section{Experimental Results}\label{Chapter:5:Results}

\textbf{Preprocessing Results:}

Preprocessing reduced a number of features from \(1.804\) to \(271\), this is a reduction at \(85\%\). 
The figure ~\ref{fig:features-preprocessing} show up the continuous change of feature amount during the preprocessing. First, the removing of near zero variance features on numerical values cut \(1117\) features. Followed by the reduction at 45 categorical features. Then, transformation process of categorical into numerical features increase the amount at \(320\), where \(22\) of them are removed in the next step by the repeated low variance analysis. Then, four features (id-number, cashflow, application-status, fraud-status) are manually removed since they are not necessary for training. As last a PCA analysis reduces the feature amount to \(271\) which are capturing \(95 \%\) of the absolute variance in given training set. Figure ~\ref{fig:pca-var} show that the first two principal components explain most of the variability in the data.

\begin{figure}[h!]
    \centering
    \includegraphics[scale=0.25]{Graphics/preprocessing-features.png}
    \caption{Feature amount changes through preprocessing steps.}
    \label{fig:features-preprocessing}
\end{figure}

\begin{figure}[h!]
    \centering
    \includegraphics[scale=0.50]{Graphics/PR-Analysis.png}
    \caption{Variance analysis of the particular principal components.}
    \label{fig:pca-var}
\end{figure}

Removing of examples contaminated by data acquisition errors reduced the sample size at \(~1\%\) to \(95.829\) examples.

Overall \(14.594.514\) categorical and \(59.224.282\) numerical feature values are imputed.


\textbf{Machine Learning Results:}

\textbf{One-Class SVM:}




% FAZIIIIIIT!!!!!






\chapter{Evaluation of Results: Value for Business}\label{Chapter:6}
Since the confusion matrices for all predictive models are available (see Chapter~\ref{Chapter:5:Results}), it is possible to analyze the potential monetary impact of a model on the micro-lending business. In this chapter, the value of a model for business is calculated.

Please note that some financial indicators used in calculations below are only approximate but in the range of real figures of a micro-lending company. For business-related reasons, the exact figures cannot be reported in this work.

The business value calculations utilize the following characteristics:

\[\textrm{(Cost of Customer Acquisition (marketing cost))} \; CAC = 0.5  \]
\[\textrm{(Mean value of the interest rate)}\; r = 10\%  \]
\[\textrm{(Mean value of a loan taken by a legitimate applicant)}\; C_p = 550  \]
\[\textrm{(Mean value of a loan taken by a fraudster)}\; C_n = 2.100  \]


First, the marketing costs for \textrm{all} negative (fraud) cases are calculated:
\[ Mcost_N = CAC * |N| \]

Once a malicious person got a loan, the company loses money. This loss is defined as:
\[ Rcost_N =   C_n * |N| \]

So, the monetary loss in case if a fraud detection model is not deployed can be calculated as follows:
\[ Loss' =  Rcost_N  +   Mcost_N  \]
\[ Loss' =  (2.100 * 743) + (0.5 * 743) \]
\[ Loss' =  1.560.671.5\]

The next step is to determine the possible impact of our best fraud detection model (PUL Ensemble without PCA and with custom voting, see the Figure~\ref{tables:confusion-matrices-ensemble} for the detailed performance result) on the calculated costs. 

The possible loss during the wrong classification of trustworthy applicants as fraudsters (missed profit here is the interest on loans) is calculated with the number of false positives:

\[ FNcost = r * |FN| * C_p = 0\]

because \[|FN| = 0.\]

Finally the gain caused by the correct classification of fraud attempts is given by:

\[Gain = C_n * |TN| = 2.100 * 542 = 1.138.200\]

The losses incurred due to wrongly issued loans to fraudsters and the marketing costs are \[2.100 * 201 + 0.5 * 201 = 422.200.5\]. Coupled with FNcost, the total loss amounts to $422.200.5$.

Thus as gains largely exceed losses when the PUL ensemble is applied, this justifies the business value of the proposed model for a given test data.

The calculations in this chapter are only preliminary and should be considered as a starting point. In the future, other business related factors, such as Customer Lifetime Value (CLF), need to be taken into account.







\chapter{Conclusion}\label{Chapter:7}

Anomalies about persons who get the loan and immediately  pay it back could not be found or it was not enough time to care about it 

Active Learning (ref to the Outlier Detection Book)

Parameter Tuning optimization, automatically optimization like hill climbing.









%\lipsum

%See also \cite{sample_bib}.
%%%%

%% appendix if used
%%\appendix
%%\typeout{===== File: appendix}
%%\include{appendix}

% bibliography and other stuff
\backmatter

\typeout{===== Section: literature}
%% read the documentation for customizing the style
\bibliographystyle{dinat}
\bibliography{sample}

\typeout{===== Section: nomenclature}
%% uncomment if a TOC entry is needed
%%\addcontentsline{toc}{chapter}{Glossar}
\renewcommand{\nomname}{Glossar}
\clearpage
\markboth{\nomname}{\nomname} %% see nomencl doc, page 9, section 4.1
\printnomenclature

%% index
\typeout{===== Section: index}
\printindex

\HAWasurency



\end{document}
